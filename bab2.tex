\chapter{KAJIAN PUSTAKA}
\section{Disesuaikan}
Kajian pustaka merupakan argumentasi ilmiah yang dipakai sebagai referensi. Bahan-bahan kajian pustaka dapat diperoleh dari berbagai sumber seperti hasil-hasil penelitian yangtelah diuji kebenarannya, jurnal penelitian, laporan penelitian,buku teks, laporan seminar, diskusi ilmiah, dan terbitan-terbitanresmi pemerintah atau lembaga-lembaga lain. Akan lebih baik jika kajian teoritis dan telaah terhadap temuan-temuan penelitian didasarkan pada sumber kepustakaan primer. Pemilihan sumber pustaka harus memenuhi dua persyaratan, yaitu:
\begin{enumerate}
\item Kemutakhiran sumber bacaan, terbitan 10 tahun terakhir dan
\item Keterkaitan antara isi bacaan dengan masalah yang dibahas.
\end{enumerate}

Langkah-langkah yang dapat dilakukan dalam kajian pustaka melalui sumber-sumber bacaan adalah sebagai berikut:
\begin{itemize}
\item[a] Memetakan konsep (concept map) keilmuan dan keislaman dengan cara mengkaji teori-teori keilmuan dan keislaman yang berhubungan dengan konsep-konsep yang dipermasalahkan dan konsep yang akan dipakai dalam analisis pembahasan.
\item[b] Membahas secara sistematis teori-teori keislaman dan keilmuan.
\item[c] Memadukan atau mengintegrasikan hasil-hasil kajian teoriyang berisi jawaban sementara (hipotesis) terhadap rumusan masalah,atau rangkuman argumentasi teoritik yang akan digunakan dalam analisis hasil kajian dengan cara mencari titik kesamaan atau perpaduan antara sains dan Islam (atau konsep yang ada pada al-Quran dan hadits).
\end{itemize}

\section{Abcdefghijk}
Kualitas hasil karya ilmiah tidak berkaitan dengan banyaknya buku yang tercantum dalam daftar rujukan, tetapi pada kualitas pustaka yang digunakan. Hal ini karena tidak jarang dijumpai skripsi yang mencantumkan daftar kepustakaan yang sangat banyak, tetapi apabila ditelusuri keterkaitan antar isi kepustakaan dan masalah yang dibahas tidak terlalu jelas.Kajian pustaka juga dapat merupakan kajian teori, yang merupakan paparan teori-teori atau konsep-konsep yang menjadi dasar pengetahuan yang diperlukan dalam análisis dan pembahasan penelitian. Definisi atau teorema yang diambil dari referensi atau buku harus mencantumkan sumbernya \cite{Liu}.

Perlu ditegaskan lagi, bahwa konsep keilmuan dan keislaman dalam kajian pustaka haruslah menjadi konsep pokok yang digunakan untuk menganalisis atau membahas masalah yang akan diselesaikan. Dengan demikian, konsep-konsep keilmuan dan keislaman yang tidak memiliki hubungan secara jelas dan tegas dengan masalah tidak perlu dicantumkan dalam kajian pustaka. Penempatan urutan kajian keilmuan dan keislaman dalam kajian pustaka disesuaikan dengan masalah penelitian \cite{Enns84,CG86,GCP89,Curtain03}.

\begin{teorema}
Untuk isi teorema, pengetikan begin dan end menggunakan bahasa Indonesia, begin{teorema}
\end{teorema}
\begin{definisi}
Untuk isi definisi, pengetikan begin dan end menggunakan bahasa Indonesia, begin{definisi}
\end{definisi}
\begin{proposisi}
Untuk isi proposisi, pengetikan begin dan end menggunakan bahasa Indonesia, begin{proposisi}
\end{proposisi}
\begin{lemma}
Untuk isi lemma, pengetikan begin dan end menggunakan bahasa Indonesia, begin{lemma}
\end{lemma}
\begin{akibat}
Untuk isi akibat, pengetikan begin dan end menggunakan bahasa Indonesia, begin{akibat}
\end{akibat}
\begin{contoh}
Untuk isi contoh, pengetikan begin dan end menggunakan bahasa Indonesia, begin{contoh}
\end{contoh}