\chapter{PEMBAHASAN}
\section{Disesuaikan}
Pembahasan untuk penelitian kepustakaan dilakukan dengan langsung menganalisis dan menguraikan konsep-konsep, keterkaitan konsep, dan pengembangannya untuk mencapai jawaban pemasalahan. Definisi-definisi dibangun dan selanjutnya dikembangkan sifat-sifat atau karakteristik baru dengan mengacu pada konsep-konsep yang dijelaskan dalam kajian pustaka.

\citeasnoun{CG86} pembahasan untuk penelitian aplikasi dimulai dengan pemaparan data dan dilanjutkan dengan analisis dan pembahasan sedangkan untuk penelitian studi kasus dimulai dengan uraian teori yang digunakan yang selanjutnya diikuti dengan data yang diperoleh untuk selanjutnya melakukan analisis kajian teori pada data.
\section{Abcdefghijk}
Kajian keislaman untuk pembahasan penelitian dapat dilakukan dalam dua cara, yaitu:
\begin{enumerate}
\item Terpadu secara langsung dalam langkah-langkah pembahasan, atau
\item Ditempatkan secara terpisah pada bagian awal atau akhir pembahasan untuk menjelaskan keterkaitan hasil penelitian dengan konsep keislaman yang ada pada kajian pustaka.
\end{enumerate}