\usepackage{colortbl}  %--- untuk bikin tabel berwarna
\usepackage{listings}  %--- untuk menulis listing program
\usepackage{xcolor}    %--- memanfaatkan warna LaTeX
\usepackage{arabtex,utf8}


\definecolor{abu}{gray}{0.80}  % = hitam 20%
\newcommand{\cca}{\cellcolor{abu}}
\newcommand{\ccb}{\bfseries}

\newcolumntype{C}{>{\centering\arraybackslash}X}


%\graphicspath{{DataGambar/}}%----folder tempat gambar-gambar---

\newcommand*{\ovc}{\textit{overcomplete}}
\newcommand*{\spn}{{\normalfont\textrm{span}}{}}
\newcommand*{\cspn}{\ol{\spn\rule{0pt}{1.3ex}}{}}
\newcommand*{\supp}{{\normalfont\textrm{supp}}}

\newcommand*{\norm}[1]{\left\|#1\right\|}          % norm
\newcommand*{\abs}[1]{\left|#1\right|}             % nilai mutlak
\newcommand*{\hkd}[1]{\left\langle#1\right\rangle} % hasil-kali-dalam
\newcommand*{\ol}[1]{\overline{#1}}                % garis atas
\newcommand*{\wh}[1]{\widehat{#1}}                 
\newcommand*{\wt}[1]{\widetilde{#1}}
\newcommand*{\ub}[1]{\underbrace{#1}}              % kurung di bawah
\newcommand*{\krk}[1]{\left\{#1\right\}}           % kurung kurawal
\newcommand*{\krb}[1]{\left(#1\right)}             % kurung biasa
\newcommand*{\krs}[1]{\left[#1\right]}             % kurung siku

\newcommand*{\mtb}[1]{\textit{\textbf{#1}}}        % bikin tulisan miring tebal 


\newtheoremstyle{teoremaku}% --- style untuk teorema dkk
    {4ex}
    {3ex}
    {\itshape}
    {}
    {\bfseries}
    {.}
    {1em}
    {}

\newtheoremstyle{contohku}% --- style untuk contoh
    {5ex}
    {3ex}
    {}
    {}
    {\bfseries}
    {.}
    {1em}
    {}

\theoremstyle{teoremaku}
\newtheorem{teorema}{Teorema}[chapter]
\newtheorem{definisi}[teorema]{Definisi}
\newtheorem{lemma}[teorema]{Lemma}
\newtheorem{proposisi}[teorema]{Proposisi}
\newtheorem{akibat}[teorema]{Akibat}

\theoremstyle{contohku}
\newtheorem{contoh}{Contoh}[chapter]

\newcommand{\refP}[1]{(\ref{#1})}
\newcommand{\refT}[1]{Teorema~\ref{#1}}
\newcommand{\refL}[1]{Lemma~\ref{#1}}
\newcommand{\refD}[1]{Definisi~\ref{#1}}
\newcommand{\refC}[1]{Contoh~\ref{#1}}
\newcommand{\refG}[1]{\ref{#1}}
\newcommand{\refTb}[1]{\ref{#1}}
\newcommand{\refS}[1]{\S\ref{#1}}

\renewcommand\theteorema{\arabic{chapter}.\arabic{teorema}}
\renewcommand\thedefinisi{\arabic{chapter}.\arabic{teorema}}
\renewcommand\thelemma{\arabic{chapter}.\arabic{teorema}}
\renewcommand\theproposisi{\arabic{chapter}.\arabic{teorema}}
\renewcommand\theakibat{\arabic{chapter}.\arabic{teorema}}
\renewcommand\thecontoh{\arabic{chapter}.\arabic{contoh}}

\newcommand{\wis}[1]{\mbox{}\hfill#1\hfill$\square$}
\newcommand{\wiss}{\mbox{}\hfill$\square$}
\newcommand{\wish}{\mbox{}\hfill$\square$}
%\renewcommand{\qedsymbol}{$\blacksquare$}
\renewcommand{\qedsymbol}{$\square$}

\hyphenation{di-de-fi-ni-si-kan mate-ma-tika multi-resolusi} 